\documentclass[]{resume}

\usepackage[style=ieee]{biblatex}
\addbibresource{ref.bib}

% 页边距:自行调整
% Word 常规:top = bottom = 2.54cm, left = right = 3.18cm
% Word 窄 :top = bottom = left = right = 1.27cm
% Word 中等:top = bottom = 2.54cm, left = right = 1.91cm
% Word 宽 :top = bottom = 2.54cm, left = right = 5.08cm
\geometry{
    a4paper,
    left=1.27cm,
    right=1.27cm,
    top=1.27cm,
    bottom=1.27cm,
    nohead,
    nofoot
}

\begin{document}

\begin{minipage}{0.82\textwidth}
    \begin{center}
        \name{姓名}
        \begin{tabular}{@{}llll}
           \faPhone      &  (+86)1xx-xxxx-xxxx  & \faEnvelope     &  \href{mailto:yourmail@qq.com}{yourmail@qq.com} \\
           \faMapMarker  &  xx省xx市            & \faBirthdayCake  & 199x \\
           \faIdCard     & 中共党员              & \faInfo         & 男 \\
           \faGithub     & \href{https://github.com/xxx}{github.com/xxx}
           % 一些可以使用的 图标:\faHome \faWeixin \faWifi \faWeibo \faQq
        \end{tabular}
    \end{center}
\end{minipage}
\begin{minipage}{3.5cm}
     \centering
     % 一寸大小:宽 * 长 = 2.5 cm * 3.5cm = 413 * 295 像素
     % 小一寸照片:宽 * 长 = 2.2 cm * 3.2 cm = 390 * 260 像素
     % 228 * 274
     \includegraphics[width=2.5cm,height=3.5cm]{./img/photo.png}
\end{minipage}

\section{\faGraduationCap \, 教育经历}

\datedsubsection{xxxx 大学 - 硕士}{20xx.9 -- 20xx.6}

\begin{tabular}{@{}llllll}
    \noindent
    学院:   & \makebox[10em][l]{xx 学院}    & 专业: & \makebox[10em][l]{xx}    & 必修课均分:&  \makebox[10em][l]{xx.xx} \\
\end{tabular}


\datedsubsection{xxxx 大学 - 本科}{20xx.9 -- 20xx.6}

\begin{tabular}{@{}llllll}
  学院:   & \makebox[10em][l]{xx 学院}  & 专业: & \makebox[10em][l]{xx}
  &
  绩点与排名: & \makebox[10em][l]{xx/xx、xx/xx} 
\end{tabular}

\section{\faAward \, 荣誉奖项}

\datedline{xx 奖学金}{20xx.x}
\datedline{xx 比赛 \quad x 等奖}{20xx.x}

\section{\faCode \, 项目与实习经历}

\subsection{项目名 1}

\begin{itemize}
    \item 内容 1
    \item 内容 2
    \item 内容 3
\end{itemize}

\datedsubsection{项目名 2}{xx 公司 \hfill 20xx.x - 20xx.x}

\begin{description}
    \item[描述] 简单介绍
    \item[内容]
        \begin{enumerate}
            \item 内容 1
            \item 内容 2
            \item 内容 3
        \end{enumerate}
    \item[成果] 成就。
\end{description}

\section{\aiGoogleScholar \, 科研成果}

\subsection{研究标题 1}

\begin{description}
    \item[描述] 主要内容
    \item[成果] \fullcite{he2016deep}. x 作,2022 JCR/IF 为 Qx/x.x。
\end{description}


\section{\faCogs \, 专业技能}

\begin{tabular}{@{}rl}
    \textbf{英语}    & CET6 \\
    \textbf{编程语言} & $\bullet$ 精通 xx。 \\
                    & $\bullet$ 熟练 xx。 \\
    \textbf{工具}    & $\bullet$ 熟悉 xx。 \\
                    & $\bullet$ 掌握 xx。 \\
    \textbf{其他}    & $\bullet$  使用 xx。\\
                    & $\bullet$ 了解 xx。 \\
\end{tabular}

\section{\faBuilding \, 校园经历}

\datedsubsection{经历名称 1}{20xx.x - 20xx.x}

\begin{itemize}
    \item 内容 1。
    \item 内容 2。
\end{itemize}

\section{\faUser \, 自我评价}

\begin{itemize}
    \item 内容 1。
    \item 内容 2。
    \item 内容 3。
\end{itemize}

\end{document}
